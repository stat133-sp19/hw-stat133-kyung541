\documentclass[]{article}
\usepackage{lmodern}
\usepackage{amssymb,amsmath}
\usepackage{ifxetex,ifluatex}
\usepackage{fixltx2e} % provides \textsubscript
\ifnum 0\ifxetex 1\fi\ifluatex 1\fi=0 % if pdftex
  \usepackage[T1]{fontenc}
  \usepackage[utf8]{inputenc}
\else % if luatex or xelatex
  \ifxetex
    \usepackage{mathspec}
  \else
    \usepackage{fontspec}
  \fi
  \defaultfontfeatures{Ligatures=TeX,Scale=MatchLowercase}
\fi
% use upquote if available, for straight quotes in verbatim environments
\IfFileExists{upquote.sty}{\usepackage{upquote}}{}
% use microtype if available
\IfFileExists{microtype.sty}{%
\usepackage{microtype}
\UseMicrotypeSet[protrusion]{basicmath} % disable protrusion for tt fonts
}{}
\usepackage[margin=1in]{geometry}
\usepackage{hyperref}
\hypersetup{unicode=true,
            pdftitle={Warmup01-Boris-Yung},
            pdfborder={0 0 0},
            breaklinks=true}
\urlstyle{same}  % don't use monospace font for urls
\usepackage{longtable,booktabs}
\usepackage{graphicx,grffile}
\makeatletter
\def\maxwidth{\ifdim\Gin@nat@width>\linewidth\linewidth\else\Gin@nat@width\fi}
\def\maxheight{\ifdim\Gin@nat@height>\textheight\textheight\else\Gin@nat@height\fi}
\makeatother
% Scale images if necessary, so that they will not overflow the page
% margins by default, and it is still possible to overwrite the defaults
% using explicit options in \includegraphics[width, height, ...]{}
\setkeys{Gin}{width=\maxwidth,height=\maxheight,keepaspectratio}
\IfFileExists{parskip.sty}{%
\usepackage{parskip}
}{% else
\setlength{\parindent}{0pt}
\setlength{\parskip}{6pt plus 2pt minus 1pt}
}
\setlength{\emergencystretch}{3em}  % prevent overfull lines
\providecommand{\tightlist}{%
  \setlength{\itemsep}{0pt}\setlength{\parskip}{0pt}}
\setcounter{secnumdepth}{0}
% Redefines (sub)paragraphs to behave more like sections
\ifx\paragraph\undefined\else
\let\oldparagraph\paragraph
\renewcommand{\paragraph}[1]{\oldparagraph{#1}\mbox{}}
\fi
\ifx\subparagraph\undefined\else
\let\oldsubparagraph\subparagraph
\renewcommand{\subparagraph}[1]{\oldsubparagraph{#1}\mbox{}}
\fi

%%% Use protect on footnotes to avoid problems with footnotes in titles
\let\rmarkdownfootnote\footnote%
\def\footnote{\protect\rmarkdownfootnote}

%%% Change title format to be more compact
\usepackage{titling}

% Create subtitle command for use in maketitle
\newcommand{\subtitle}[1]{
  \posttitle{
    \begin{center}\large#1\end{center}
    }
}

\setlength{\droptitle}{-2em}

  \title{Warmup01-Boris-Yung}
    \pretitle{\vspace{\droptitle}\centering\huge}
  \posttitle{\par}
    \author{}
    \preauthor{}\postauthor{}
    \date{}
    \predate{}\postdate{}
  
\usepackage{bbm}

\begin{document}
\maketitle

\section{\texorpdfstring{\textbf{Star
Wars}}{Star Wars}}\label{star-wars}

\subsubsection{\texorpdfstring{\emph{Kazuda Xia
(Kaz)}}{Kazuda Xia (Kaz)}}\label{kazuda-xia-kaz}

\begin{figure}
\centering
\includegraphics{https://starwars.fandom.com/wiki/Kazuda_Xiono?file=Kazuda.jpg}
\caption{Kazuda Xia}
\end{figure}

A famous quote from Kazuda Xiono:

\begin{quote}
In my mind, that is what I was- Kazuda Xiono, the best starfighter pilot
in the galaxy.
\end{quote}

\begin{longtable}[]{@{}ll@{}}
\toprule
Description & Value\tabularnewline
\midrule
\endhead
Species: & Human\tabularnewline
Gender: & Male\tabularnewline
Hair Color: & Black\tabularnewline
Eye Color: & Brown\tabularnewline
Skin Color: & Light\tabularnewline
\bottomrule
\end{longtable}

\begin{center}\rule{0.5\linewidth}{\linethickness}\end{center}

\section{\texorpdfstring{\textbf{Cooking
Recipe}}{Cooking Recipe}}\label{cooking-recipe}

\subsubsection{\texorpdfstring{\emph{Baked Buffalo
Wings}}{Baked Buffalo Wings}}\label{baked-buffalo-wings}

\subparagraph{ingredients}\label{ingredients}

\begin{itemize}
\tightlist
\item
  1 to 4 pounds chicken wings that have been cut into flats and
  drumettes.
\item
  1 1/2 teaspoons to 2 tablespoons baking powder (using 1 1/2 teaspoons
  per pound).
\item
  1 to 4 teaspoons kosher salt (using 1 teaspoon per pound; use less
  salt if you're not using Diamond brand).
\item
  2 tablespoons hot sauce (Frank's RedHot is traditional) per pound of
  wings.
\item
  1 1/2 to 2 tablespoons unsalted butter per pound of wings.
\item
  To serve (optional): Homemade Blue Cheese Dressing, plus celery and
  carrot sticks.
\end{itemize}

\subparagraph{Special kitchen tools}\label{special-kitchen-tools}

\begin{itemize}
\tightlist
\item
  Large baking shhet with foil
\item
  oven-safe cooling rack
\end{itemize}

\subparagraph{Steps}\label{steps}

\begin{enumerate}
\def\labelenumi{\arabic{enumi}.}
\tightlist
\item
  Pat wings dry with a paper towel
\item
  Toss them with 1 1/2 teaspoons baking powder and 1 teaspoon kosher
  salt per pound of wings in a large bowl until thoroughly coated.
\item
  Arrange them on the rack with some space between them (the surface
  will not dry and crisp as well where they touch) and place in your
  refrigerator uncovered for 8 to 24 hours.
\item
  Heat your oven to 450 degrees, with a rack set in the top half of your
  oven.
\item
  Bake your wings for 20 minutes
\item
  Flip them with tongs or a spatula, bake them for another 15 minutes
\item
  Flip them back over again, and then for 15 final minutes, for a total
  roasting time of 50 minutes.
\end{enumerate}

(Addidtion) You might need up to 5 minutes longer for larger wings.
Wings are done with they are browned and crisp.

\subparagraph{Source}\label{source}

In a medium saucepan, for each \textbf{pound} of wings

Combine 2 tablespoons of hot sauce and 1 1/2 tablespoons (for a hotter
sauce) to 2 tablespoons (for a mild, more buttery sauce) unsalted butter
over medium heat until melted, whisking to combine. Set aside.

\emph{There is no special season of the year to enjoy buffalo wings. You
can enjoy buffalo wings ay any time you wants.}

\emph{There are no any other recipe for bake buffalo wings. If you want
to have try out different wings, there are pleanty of recipe that can
teach you how to make other kind of wings!!!}

\begin{center}\rule{0.5\linewidth}{\linethickness}\end{center}

\section{\texorpdfstring{\textbf{Euclidean
Distance}}{Euclidean Distance}}\label{euclidean-distance}

\subsection{\texorpdfstring{\#\#\#Definition \href{}{{[}edit{]}}
\#\#\#}{\#\#\#Definition {[}edit{]} \#\#\#}}\label{definition-edit}

The \textbf{Eucliden distance} between points \textbf{p} and \textbf{q}
is the length of the \href{}{line segment} connecting them.
(\overline{**pq**}).

In Cartesian coordinates(), if
\[\boldsymbol{p} = (P_{1}, P_{2},\dotsc,P_{n})\] and \boldsymbol{q} =
are two points in \href{}{Euclidean \emph{n-}space}, then the distance
9d0 from \textbf{p} to \textbf{q}, or from \textbf{q} to \textbf{p} is
given by the Pythagorean formula.

d(\textbf{p},\textbf{q}) = d(\textbf{q},\textbf{p}) =
\sqrt{(q_{1}- p_{1})^2 + (q_{2}- p_{2})^2 + \dots + q_{n}-p_{n}^2} =
\sqrt{\displaystyle\sum_{i=1}^{10} (q_{i}-p_{i})^2}

The posotion of a point in a Euclidean \emph{n}-space is a
\href{}{Euclidean vector}. So,\textbf{p} and \textbf{q} may be
represented as Euclidean vectors, starting from the origin of the space
(initial point) with ther tips ( terminal points) ending at the two
points. The \href{}{Euclidean norm}, or \textbf{Euclidean length} or
\textbf{magnitude} of a vecttor measurethe length of he vector:

\Vmattrix{P} = \sqrt{P_1^2 + P_2^2 + \dots + P_n^2} =
\sqrt {**P** . **p**},

wherre the last expression involves the \href{}{dot product}.


\end{document}
